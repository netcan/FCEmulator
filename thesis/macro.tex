% 模板设置

\newcommand{\privacy}[1][绝密$\star$启用前]{#1} % 密级
\newcommand{\type}[1][设计]{#1} % 类型
\newcommand{\titleCn}[1][基于6502 cpu的NES模拟器设计与实现]{#1} % 中文题目
\newcommand{\titleEn}[1][Design and Implementation of a NES Emulator Based on 6502 cpu]{#1} % 英文题目

\newcommand{\keywordsCn}[1][模拟器;6502;计算机组成原理;NES]{#1} % 中文关键字
\newcommand{\keywordsEn}[1][Emulator; 6502; Computer Organization; NES]{#1} % 英文关键字

\newcommand{\supervisor}[1][安鑫-副教授]{#1} % 导师姓名
\newcommand{\studentID}[1][2014218760]{#1} % 学号
\newcommand{\studentNameCn}[1][罗能]{#1} % 填写中文姓名
\newcommand{\studentNameEn}[1][Neng Luo]{#1} % 填写英文姓名

\newcommand{\finishedYear}[1][\the\year]{#1} % 论文完成日期: 年
\newcommand{\finishedMonth}[1][\the\month]{#1} % 论文完成日期: 月
\newcommand{\finishedDay}[1][\the\day]{#1} % 论文完成日期: 日

\newcommand{\department}[1][信息工程系]{#1} % 系名称
\newcommand{\major}[1][计算机科学与技术]{#1} % 专业名称
\newcommand{\enrolmentYear}[1][2014级]{#1} % 入学年份



% 字号设置
\newcommand{\chuhao}{\fontsize{42pt}{\baselineskip}\selectfont}     % 字号设置
\newcommand{\xiaochuhao}{\fontsize{36pt}{\baselineskip}\selectfont} % 字号设置
\newcommand{\yichu}{\fontsize{32pt}{\baselineskip}\selectfont}      % 字号设置
\newcommand{\yihao}{\fontsize{28pt}{\baselineskip}\selectfont}      % 字号设置
\newcommand{\erhao}{\fontsize{21pt}{\baselineskip}\selectfont}      % 字号设置
\newcommand{\xiaoerhao}{\fontsize{18pt}{\baselineskip}\selectfont}  % 字号设置
\newcommand{\sanhao}{\fontsize{15.75pt}{\baselineskip}\selectfont}  % 字号设置
\newcommand{\sihao}{\fontsize{14pt}{\baselineskip}\selectfont}      % 字号设置
\newcommand{\xiaosihao}{\fontsize{12pt}{\baselineskip}\selectfont}  % 字号设置
\newcommand{\wuhao}{\fontsize{10.5pt}{\baselineskip}\selectfont}    % 字号设置
\newcommand{\xiaowuhao}{\fontsize{9pt}{\baselineskip}\selectfont}   % 字号设置
\newcommand{\liuhao}{\fontsize{7.875pt}{\baselineskip}\selectfont}  % 字号设置
\newcommand{\qihao}{\fontsize{5.25pt}{\baselineskip}\selectfont}    % 字号设置

% 下划线
\newcommand{\underlineFixlen}[2][3.5cm]{\underline{\makebox[#1][c]{#2}}}


% 中文摘要
\renewenvironment{abstract}{
\thispagestyle{empty} % 去掉页码
{
\begin{center}
\Large \songti \bfseries 摘\hspace{1em}要\vspace{1.1cm}
\end{center}
}
\setlength{\parindent}{2em}
\setlength{\parskip}{0em}
\setlength{\baselineskip}{22pt} % (宋体,小四;固定行距22磅,段前、段后均为0行间距。段落首行缩进2字符。)
\songti
}{
\setlength{\parindent}{0em}
\setlength{\parskip}{1em}
{\par \songti \bfseries{关键词:}}
\keywordsCn
\clearpage
}

% 英文摘要
\newenvironment{abstractEn}{
\thispagestyle{empty} % 去掉页码
{
\begin{center}
\Large \bfseries ABSTRACT\vspace{1.5cm}
\end{center}
}
\setlength{\parindent}{1em}
\setlength{\parskip}{0em}
\setlength{\baselineskip}{22pt} % 22磅行距,首行缩进1字符,段前、段后均为0行间距
}{
\setlength{\parindent}{0em}
\setlength{\parskip}{1em}
{\par \bfseries{KEYWORDS:}}
\keywordsEn
\clearpage
}

% 目录名
\renewcommand\contentsname{
\begin{center}
\songti \Large \bfseries 目\hspace{1em}录 % (宋体,小二号,加粗;居中,单倍行距,段前0.5行、段后1.5行间距)
\end{center}
\vspace{1em}
}
% 插图清单
\renewcommand\listfigurename{
\begin{center}
\songti \Large \bfseries 插图清单 % (宋体,小二号,加粗;居中,单倍行距,段前0.5行、段后1.5行间距)
\end{center}
\vspace{1em}
}

% 表格清单
\renewcommand\listtablename{
\begin{center}
\songti \Large \bfseries 表格清单 % (宋体,小二号,加粗;居中,单倍行距,段前0.5行、段后1.5行间距)
\end{center}
\vspace{1em}
}

\renewcommand\refname{\heiti \sanhao \bfseries 参考文献}


% 目录引线设置
\renewcommand{\cftdotsep}{1.5} % 线的密度
\renewcommand{\cftsecdotsep}{1.5} % section引线
\renewcommand{\cftsecleader}{\cftdotfill{\cftsecdotsep}}
\renewcommand{\cftsecpagefont}{}

% 插图清单
\renewcommand{\cftfigpresnum}{\figurename\enspace}

% 表格清单
\renewcommand{\cfttabpresnum}{\tablename\enspace}

% 致谢
\newenvironment{acknowledge}{
\clearpage
\vspace*{-2em}
\phantomsection % 使得hyperref目录能够跳转到正确的位置
\addcontentsline{toc}{section}{致谢} % 添加到目录中
\begin{center}
 \songti \Large \bfseries 致谢\end{center}\vspace{1.1cm}
\setlength{\parindent}{2em}
\setlength{\parskip}{0.5em}
\setlength{\baselineskip}{22pt} % 22磅行距,首行缩进1字符,段前、段后均为0行间距
\songti
\par
}{
	\par
	\hfill 作者:\studentNameCn

	\hfill \finishedYear\enspace 年\finishedMonth\enspace 月\finishedDay\enspace 日
}

% 附录
\renewenvironment{appendix}{
\clearpage
\vspace*{-2em}
\phantomsection % 使得hyperref目录能够跳转到正确的位置
\addcontentsline{toc}{section}{附录} % 添加到目录中
\begin{center}
 \songti \Large \bfseries 附录\end{center}\vspace{1.1cm}
\setlength{\parindent}{2em}
\setlength{\parskip}{0.5em}
\setlength{\baselineskip}{22pt} % 22磅行距,首行缩进1字符,段前、段后均为0行间距
\songti
\par
}{
}

% 图名称
\renewcommand{\figurename}{图}
\renewcommand{\tablename}{表}

% 无序列表
\renewcommand{\labelitemi}{$\bullet$}
\renewcommand{\labelitemii}{$\circ$}
